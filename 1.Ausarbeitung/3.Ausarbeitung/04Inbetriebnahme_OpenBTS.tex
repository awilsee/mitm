
\section{Inbetriebnahme eines OpenBTS Systems}
Für Inbetriebnahme des GSM Netzes waren einige Vorinstallationen sowie das Einrichten von Ubuntu 16.04.3 nötig. Im Folgenden wird das Vorgehen zur Einrichtung des Systems sowie die Inbetriebnahme des GSM Netzes beschrieben.

\subsection{Vorinstallationen}

\subsubsection{Ubuntu 16.04.3}
Zunächst wurde wie bei der Installation von Osmocom das Betriebssystem Ubuntu 16.04.3 auf einem Labor-Rechner installiert und eingerichtet.

\subsubsection{Git}
Auch Range Networks nutzt für OpenBTS Git-Repositories, weshalb auch für die Installation von OpenBTS zunächst Git eingerichtet wurde. Zur Versionskontrolle und Verwaltung des Codes wurde auch hier das Team-interne Git Repository genutzt.

\begin{lstlisting}
sudo apt-get install git
\end{lstlisting}

\subsubsection{Softwarevoraussetzungen}
Um OpenBTS fehlerfrei installieren und in Betrieb nehmen zu können, wurden zunächst einige Bibliotheken und Pakete via Paketmanager installiert, um die vorausgesetzten Abhängigkeiten zu erfüllen:
\begin{lstlisting}
sudo apt-get install autoconf libtool libosip2-dev libortp-dev libusb-1.0-0-dev g++ sqlite3 libsqlite3-dev erlang libreadline6-dev libncurses5-dev
\end{lstlisting}

\subsubsection{Aktivierung der Verbindung zum USRP2}
Nachdem wir in Kapitel~\ref{usrp2_connect} die IP-Adresse des N210 erfolgreich zurücksetzen konnten, mussten wir in diesem Fall nur eine Verbindung zum USRP2-Gerät aufnehmen, indem wir die IP-Adresse des PCs innerhalb des gleichen Netzwerkbereichs setzten.

\begin{lstlisting}
sudo ifconfig enp0s25 192.168.10.3
\end{lstlisting}

Dieses mal konnten wir mithilfe der automatischen Erkennungsfunktion direkt erkennen, dass eine Verbindung zum N210 bestand.
\begin{lstlisting}
uhd_find_devices
\end{lstlisting}

Da uns im Laufe des Projekts die IP-Adresse der Schnittstelle am PC immer wieder zurückgesetzt wurde, speicherten wir die IP-Adresse daraufhin direkt in den Netzwerkeinstellungen.

\subsection{Installation einzelner GSM Komponenten}\label{GSM_Komp}
Range Networks stellt eine detaillierte Anleitung zur Installation von OpenBTS inklusive aller zuvor beschriebenen Komponenten bereit, die zur Inbetriebnahme eines GSM-Netzes benötigt werden. Im Folgenden werden die Schritte zur Umsetzung auf der uns vorgelegenen Hardware genauer beschrieben.

\subsection{Starten des Systems}
