\section{Einleitung}
Im Rahmen der Vorlesung Mobile Netze soll zunächst ein Mobilfunknetz in Betrieb genommen werden. Des Weiteren umfasst die Aufgabenstellung die Implementierung eines Features, welches vom Team in einer Projektvision definiert wird. \\

Als Projektziel des Teams J3A soll eine Man-In-The-Middle Funktionalität in einem GSM Netz integriert werden. Man-In-The-Middle bezeichnet eine Angriffsform, bei der ein Dritter den Datenverkehr zweier Gesprächspartner mitverfolgt. Dabei bleibt er unbemerkt und ist in der Lage alle Daten, die über die Kommunikationsverbindung gesendet werden, abzugreifen. Oft täuscht er den eigentlichen Gesprächspartner vor, sodass kein Verdacht geschöpft wird.\\

Der Fokus des Projekts ähnelt stark dem Prinzip des Man-In-The-Middle Angriffs. Das Ziel ist es ein Telefongespräch abzugreifen und abzuspeichern, der zwischen zwei Teilnehmern getätigt wird. Dabei sollen die abgespeicherten Daten in eine Form gebracht werden, die das Anhören lokal auf dem Rechner ermöglicht. \\

Die Umsetzung der Aufgabenstellung umfasst die Inbetriebnahme des GSM Netzes, das aus den Komponenten BTS, BSC und MSC besteht. Über Voice over IP soll der zweite Teilnehmer erreicht werden können. Alternativ zu Voice over IP war anfänglich auch die Verbindung zum zweiten Teilnehmer ebenfalls im GSM denkbar. Die minimale Anforderung besteht in der Manipulation der BTS/BSC dahingehend, dass diese die Daten abgreift und abspeichert. Das Hinterlegen einer Rufnummer, unter der der gespeicherte Anruf angehört werden kann, wurde als optionales Feature bezüglich des Projektzieles definiert.
