\section{Ergebnisse} 
Insgesamt lässt sich sagen, dass die definierten Projektziele in vollem Umfang erreicht wurden. Die Inbetriebnahme des GSM Netzes wurde erfolgreich umgesetzt. Dabei wurden zwei unterschiedliche Architekturen parallel verfolgt, sodass letztendlich zwei verschiedene Netze in Betrieb genommen werden konnten. Aufgrund der geringen Einarbeitungszeit und der begrenzten Projektdauer konnte auf diese Weise sichergestellt werden, dass ein funktionierendes System zur weiteren Verfolgung des Projektzieles zur Verfügung stand. Des Weiteren wurde die Funktionalität des Abgreifens und Abspeicherns eines Telefongesprächs vollständig realisiert. Dabei werden die SIP/RTP Sessions durch das Tool pcapsipdump aufgezeichnet und im .pcap Format gespeichert. Anschließend wird die .pcap Datei in eine .gsm Datei und in das einfacher abspielbare .wav Format konvertiert. Dieses Vorgehen wurde schließlich automatisiert, sodass nach einem getätigten Anruf im Testnetz eine Aufnahme des Gesprächs lokal auf dem Rechner abgespielt werden kann.\\

Außerdem wurde im Projektziel ein optionales Feature formuliert, das ebenfalls vollständig umgesetzt werden konnte. Dieses umfasst die Hinterlegung einer bzw mehreren Rufnummern und das Abspielen des aufgezeichneten Telefongespräches bei Anruf dieser Nummern. Es wurde pro Telefonnummer ein Slot erstellt, sodass immer die letzten Gespräche aller registrierten Teilnehmer abgehört werden können. Zusätzlich wurde nur eine festgelegte IMSI authentifiziert die Gespräche wieder abzuhören, sodass nicht jeder, wer jetzt die Nummern anruft, einfach die Gespräche abhören kann.\\

Beim Vergleich der beiden Architekturen lässt sich sagen, dass bei beiden zunächst einiges installiert und konfiguriert werden musste. Bei Osmocom gab es häufiger das Phänomen, dass beim Beenden eines Anrufs es länger dauerte bis der Anruf dann auch wirklich beendet wurde, sodass die Reaktionszeit bei OpenBTS geringer war. Zudem war die Sprachqualität ebenfalls bei OpenBTS besser bzw klarer. Jedoch ist Osmocom mit den vielen einzelnen Bausteinen, welche auch der realen Welt entsprechen, flexibler, da bei OpenBTS vieles in dem Hauptmodul ist.
Allerdings gibt es keinen klaren Gewinner oder Verlierer.

\section{Fazit}
Durch die effektive Zusammenarbeit sowie team-interne Aufgabenverteilung ergab sich ein rasches Vorankommen. Nichtsdestotrotz ergaben sich Probleme bei der Installation einzelner GSM Komponenten. Das Installieren und Konfigurieren zusätzlicher Abhängigkeiten erschwerte die Inbetriebnahme. So war anfänglich unklar, wie die Konfigurationsdateien von OpenBSC und OsmoBTS auf unsere Anforderungen angepasst werden mussten. Hinzu kamen Probleme beim Kompilieren dieser Dateien. Zur Umsetzung des Projektziels mussten zunächst Recherchen angestellt werden. Da die Vorgehensweise nicht immer eindeutig war, wurden zusätzliche Aufwände für das Ausprobieren mehrerer Möglichkeiten aufgebracht. Alles in allem wurde das Projektziel für alle Beteiligten zur vollkommenen Zufriedenheit zeitgerecht umgesetzt.


