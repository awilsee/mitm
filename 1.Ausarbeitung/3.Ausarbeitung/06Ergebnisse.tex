\section{Ergebnisse} 
Insgesamt lässt sich sagen, dass die definierten Projektziele in vollem Umfang erreicht wurden. Die Inbetriebnahme des GSM Netzes wurde erfolgreich umgesetzt. Dabei wurden zwei unterschiedliche Architekturen parallel verfolgt, sodass letztendlich zwei verschiedene Netze in Betrieb genommen werden konnten. Aufgrund der geringen Einarbeitungszeit und der begrenzten Projektdauer konnte auf diese Weise sichergestellt werden, dass ein funktionierendes System zur weiteren Verfolgung des Projektzieles zur Verfügung stand. Des Weiteren wurde die Funktionalität des Abgreifens und Abspeicherns eines Telefongesprächs vollständig realisiert. Dabei werden die SIP/RTP Sessions durch das Tool pcapsipdump aufgezeichnet und im .pcap Format gespeichert. Anschließend wird die .pcap Datei zunächst in eine .gsm Datei und anschließend in das abspielbare .wav Format konvertiert. Dieses Vorgehen wurde schließlich automatisiert, sodass nach einem getätigten Anruf im Testnetz eine Aufnahme des Gesprächs lokal auf dem Rechner abgespielt werden kann.

\section{Fazit}
Durch die effektive Zusammenarbeit sowie team-interne Aufgabenverteilung ergab sich ein rasches Vorankommen. Nichtsdestotrotz ergaben sich Probleme bei der Installation einzelner GSM Komponenten. Das Installieren und Konfigurieren zusätzlicher Abhängigkeiten erschwerte die Inbetriebnahme. So war anfänglich unklar, wie die Konfigurationsdateien von OpenBSC und OsmoBTS auf unsere Anforderungen angepasst werden mussten. Hinzu kamen Probleme beim Kompilieren dieser Dateien. 


